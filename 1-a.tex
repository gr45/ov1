To recap, the system is intended to serve as everyone's go-to service when they need to travel from some place in Norway to another place in Norway, when the distance is too great to be considered within ``walking distance''.

Simply put, our system makes planning one's travels easier.
Life is an optimization problem. Anything that simplifies parts of it or frees up our time is arguably a solution to the problem of ``not having enough time''.
More specifically, our system aims to solve the following problems that exist with the current model:
\begin{enumerate}
	\item{Proficient use of public transportation in some areas seem to require intricate arcane knowledge only possessed by those native to the area.}
	\item{Information about available methods of transportation in an area is not readily available unless one knows where to look.}
	\item{Current systems are segmented according to provider with little or no support for planning a trip outside any single provider's reach.}
\end{enumerate}
The first issue is solved as the system provides easy-to-understand information in the same format wherever our users are.
People who have moved to or visited some place in Norway they've not been before might be familiar with this.
The second issue is solved by offering this information through the system, regardless of where the user might be located.
The third issue, which you might be familiar with if you've ever flown out of Trondheim\footnote{Where there's two different companies running a bus route to the airport, each listing their routes at different locations and of course neither are compatible with the regular bus system.},
is also solved by the collecting of all this information in one database, allowing the system to search through all of it when looking for a viable travel route.

The key thing here is that all this information -- in-land flight schedules, bus routes, train departures, trams, ferry schedules throughout the country -- is all gathered up into one place, lowering the entry level for those who want to use non-private transportation.
The current system is like having multiple dictionaries, with the words spread randomly between them. Why would anyone want that?

The system itself, as presented to the user, is in its most simple form just two fields -- into which the user enters his or hers place of departure and desired destination -- and a button that is pushed after the information is entered.
A suitable route is then computed by the system and presented to the user.
The route is presented in the form of a map, as well as a series of concisely formulated steps the user would have to undertake in order to reach his or her destination.
This could be an app, a website or a telephone-based service. The core concept is that the user provides the system with a place of departure and desired destination, and is given a route to get there which employs as much non-private transportation as possible.	