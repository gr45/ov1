There is a variety of challenges facing a product such as a travel planner. The
most prominent of which is the development of proper APIs. As of yet there are
many different services available in various areas that provides information on
travel planning in that area. As these services are commonly available on the
web, the travel planner will be able to extract information from these when
appropriate. Unfortunately this is not standardized, and all of these services 
will have to be uncovered manually.

The amount of data that will be processed and gathered at any step to plan a trip will be within the
scope of what current technology is able to. Therefore the technological
challenges from a hardware perspective will be negligible. It is the software's
decision making on where it should gather data for what are that will be
important. The software that handles this could have a feed-back stage so that
the user can respond whether they believe the travel planner has presented a
good route/method of travelling. If it is unacceptable the travel planner may learn
from its mistakes and perform better next time. How the travel planner will
handle these issues depends on what the expected demographic and userbase will
be. If the userbase is small and exclusive, the travel planner may prioritize
quality over efficiency or the urban over the suburban. This means that the
travel planner will require a developmental stage that can take a significant
amount of time and resources, but once it is up and running, the system will be
relatively self-sufficient.

Scaling as a result of popularity might become an
issue when regarding the servers. This is a problem that has been faced by many
up-and-coming web based businesses. Some have handled this by preparing the
community for a downtime, others have tried to perform the transition in the
background. 
