\subsection{Possible Business Models}

There are primarily two ways of monetizing a product such as ours: Selling the
product to our end users, or selling our end users to other companies. The
former would typically involve a subscription fee, which would require users
to authenticate themselves prior to using the service, and/or charging a
one-time fee for our mobile apps. Selling our end users might mean showing
advertisements in our trip planner, or taking a commission on the sale of
tickets.

Each of these models have their pros and cons. Charging customers to use our
service is the most straight-forward way of making a profit, allowing us to
focus on making the planner as good as possible, but has the disadvantage
of presenting a barrier of entry for potential customers. It's especially
problematic to charge for a service such as a trip planner because it is
not common to do so; many people would likely look to a free alternative,
even if it was not particularly good, making it difficult for us to achieve
the critical mass of users necessary to consolidate a segregated market.

Showing advertisments is problematic because ``nobody likes ads'', but has the
advantage of keeping the service free. It also keeps business negotiations
simple provided we utilize large ad publishing programs such as those offered
by Google and Apple. Taking a commission on tickets sold by transportation
service providers is problematic because it would compromise our integrity if
companies were able to pay for a higher position in our search results, or
our perceived integrity if people beleived that this was possible. Taking a
commission also requires deals to be made with various companies, increasing
the administrative/burocratic overhead of our operation, an often-unpopular
move with start-ups in technology.

Of these possibilities, displaying advertisements alongside our trip routes
would probably be the path of least resistance to making a profit.
