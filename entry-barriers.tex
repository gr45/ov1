\subsection{Entry barriers}
Trip planners aren't a new invention by any standard, and there exist a lot of alternatives like Ruter, AtB, Skyss and so on. The problem with the existing trip planners are that they are only made for specific regions.
Because there already exists different trip planners we can learn from them and improve on their features. This will make it easier to come up with a better system that can deliver a better result.
A trip planner will typically be implemented as a website complimented with a mobile app for mobile devices.
Websites and mobile apps are relatively easy to make, which means that the entry barrier is low for development of the product and also easy to use for customers that want to try out the system.

As a new player in the market, it is important that people be made aware of the new product. A lot of resources will be needed for a nationwide campaign, which is required as our potential customers include everyone in the country.
Another difficulty is attaining credibility; to gain the trust of the people.
Most public transportation companies have their own trip planner. It is important that the customer can trust that the information provided by our trip planner is just as good, or better, as the information provided by existing trip planners.
It is also essential that the route being calculated is optimal to the wishes of the customer. 