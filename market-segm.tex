\subsection{Market segmentation and quantifying potential market size}
In general we can divide the market for our trip planner into three groups:
\begin{enumerate}
	\item{Those who are going to use the trip planner a lot; people who often travel to new places.}
	\item{Those who will use the trip planner some times; these people do not travel very much or generally just travel between a small set of the same places.}
	\item{The last group is the tourists coming from other countries.}
\end{enumerate}
These groups are further divisible into other groups as well. Some people are willing to pay more for their trip duration to be shorter. This is generally people with a lot of money and people traveling at their employers expense. Other people may want the trip to be as cheap as possible. This could be people with very limited funds, e.g. students.
Most people will probably want a balance between cost and travel time.

For now we are looking into a nationwide trip planner, connecting other existing regional trip planners into one functional product. This will make it easier to plan trips within different regions and in between different regions.

The target group of customers will be those traveling domestic. This can be both norwegian citizens as well as foreign tourists. This will help them plan the trip both cost- and time wise.
As most people traveling domestic will, at some point, whether it is a long trip or just a short trip, need to look up the time table(s) and plan their trip, this service will be needed by everybody.
Our potential customer base is therefore everyone traveling by bus, train, tram, boat, plane and all other methods of transportation available in Norway. Essentially, this includes everyone.
Even if one person is only traveling the same distance time after time after time, he/she will have to look up an updated schedule from time to time.
Maybe there have been changes to the time table or another alternative has become available.