oppgave1 + a-1

Idea 1)
``Gløshaugen StandGuide''
Students at the Norwegian University of Technology and Science, 
especially those belonging to the Gløshaugen campus, 
become accustomed to the somewhat sporadic appearance of ``stands'', 
from freebies such as coffee can be obtained, on and about the campus.
This is universally perceived as a positive phenomenon by students.
In fact, the only downside to this arrangement seems to be that the knowledge of the presence of such stands are spread through word-of-mouth and/or the grapevine,
meaning individuals have to run around campus looking for stands.
The year is 2013 -- the world is more modern than ever before, 
is such an archaic approach really acceptable?
Introducing ``Gløshaugen StandGuide'', a system that lets you see if there's any stands on campus at any given time and what they're offering.
No more will students have to run around like headless chickens uncertain as to whether they'll find any free coffee by the end of their run or not.
It's efficient, it's attractive, it's easy.
It's the future.
``Gløshaugen StandGuide'' -- because you shouldn't have to work for free (stuff).


Travel planner.
People travel. By car, foot, train, plane, bicycle, 
taxi or even multidimensional transportalization devices --
regardless of preferred method of travel, people move about. 
But how do they know where to go? Or rather, how to get there?
Very few people get through life without ever having to get from some specific place to another place for the first time.
Way back in the distant past, people were pretty much out of luck if they ever found themselves in such a precarious situation.
But these days, readily available technologies make it super easy to figure out how to get from a to b, even if you've never been to either before.
Just hit up google maps or whatevs and bam! It'll plot out a path for you.
Easy as lemon pie.
But what if you don't have a car? What if you don't have a bike, or th etime to walk?
What if you have to take the bus but don't know 	
Way back in the distant past, people had little choice but to walk.

How d
At some point in your life you might have found yourself in a situation where you were at some place, 
A, and had to get to another place, B.



You might have at some point found yourself in a situation where you're at some place, A, and need to get to another place, B.



an archaic method 
Wouldn't it be 
(practically overflowing)