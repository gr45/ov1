oppgave1 + a-1

Idea 1)
``Gløshaugen StandGuide''
Students at the Norwegian University of Technology and Science, 
especially those belonging to the Gløshaugen campus, 
become accustomed to the somewhat sporadic appearance of ``stands'', 
from freebies such as coffee can be obtained, on and about the campus.
This is universally perceived as a positive phenomenon by students.
In fact, the only downside to this arrangement seems to be that the knowledge of the presence of such stands are spread through word-of-mouth and/or the grapevine,
meaning individuals have to run around campus looking for stands.
The year is 2013 -- the world is more modern than ever before, 
is such an archaic approach really acceptable?
Introducing ``Gløshaugen StandGuide'', a system that lets you see if there's any stands on campus at any given time and what they're offering.
No more will students have to run around like headless chickens uncertain as to whether they'll find any free coffee by the end of their run or not.
It's efficient, it's attractive, it's easy.
It's the future.
``Gløshaugen StandGuide'' -- because you shouldn't have to work for free (stuff).

Idea 2)
Travel planning system for transportation within Norway.
``Reiseplanlegger1''

If you want to travel within Norway your options are quite plentiful.
There's trains, buses, metros, trams, airplanes and what not.
Norway's problem isn't that it lacks alternatives to private travel methods such as the personal car, 
it's that figuring out how to use these alternatives can be a soul-draining experience.
The company running all public transportation in one city has its own system,
which is wholly different from whatever system is in use in any other city.
Knowing how to figure out what bus you need to take to get somewhere in Oslo doesn't really help you pick out the right bus in Trondheim, and vice versa.
Travelling from somewhere in Oslo to somewhere in Trondheim requires that you interface with up to five\footnote{1) public transit to get to the airport transit, 2) transit to oslo airport, 3) airline, 4) airport transit from trondheim airport, 5) public transit in Trondheim} different providers of transportation just to plan your trip.

It shouldn't have to be that complicated.
That's why we've come up with ``Reiseplanlegger1'',
a system which aims to integrate every single method of publicly available transportation within Norway, 
allowing you to plan your trip through one single interface. 
Need to get from Skånevik to Molde?
If it's possible, we'll tell you how.